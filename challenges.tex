\chapter{Des stratégies qui paient et des challenges à venir}

En supplément du Chapitre 3, il est à remarquer que les quatre entreprises étudiées ont suivi la \textit{Stratégie Océan Bleu}\supercite{BlueOceanStrategy}.

Dans cet ouvrage écrit par deux chercheurs de l'\textit{INSEAD}, le marché se compose de deux ensembles d'océans : les \textbf{océans rouges}, la partie qui est déjà exploitée par des entreprises ; et les \textbf{océans bleus}, la partie du marché qui est encore vierge de toute activité et concurrence.

L'ouvrage préconise aux entreprises de \textbf{conquérir ces océans bleus incontestés} en y créant une demande et de \textbf{profiter de l'absence de concurrence établie pour croître rapidement}.

\vspace{5mm}

Ainsi SpaceX s'est d'abord concentré sur le lancement de petits satellites avant de s'atteler aux marchés plus important, 

Tesla Motors a créé une voiture entièrement électrique à une époque où les constructeurs se limitaient aux véhicules hybrides, 

Spotify a été le premier service d'écoute de musique à la demande crédible\footnote{De nombreuses radios en ligne (dont \textit{Pandora}) et \textit{Blogmusik} (l'ancêtre de \textit{Deezer}) existaient déjà lors du lancement de la bêta de Spotify en 2007. Bien que Blogmusik est considéré comme le premier service d'écoute de musique à la demande, il opérait en toute illégalité.} et fut lancé alors que \textit{The Pirate Bay} était à son apogée,

et Netflix a été le premier sur la location de DVD et la vidéo à la demande illimitée en Amérique du Nord.

\vspace{5mm}

D'autant la Stratégie de l'Océan Bleu est extrêmement pratique pour s'insérer dans un marché et croître rapidement, il faut savoir adopter une stratégie à plus long terme afin de pouvoir \textbf{continuer à croître} et \textbf{entrer en compétition avec les concurrents établis}.

\vspace{5mm}

Après que SpaceX ait pu obtenir des \textbf{revenus réguliers grâce à ses nombreux lancements}, l'entreprise a pu sereinement lancer le développement de la fusée Falcon 9 et de la capsule Dragon. Ces deux développements leur ont alors permis d'ouvrir les portes des contrats gouvernementaux avec la NASA pour \textbf{ravitailler la Station Spatiale Internationale} et de \textbf{rentrer en compétition} directe avec \textit{Arianespace} et \textit{RSC Energia} qui avaient récupéré la majeure partie du marché après l'abandon du programme \textit{Space Shuttle}.

Arianespace est alors \textbf{mis en difficulté} puisqu'un lancement d'Ariane 5 coûte entre 165 et 220 millions de dollars\supercite{Ariane5Cost} tandis qu'un lancement de Falcon 9 ne coûte que 62 millions de dollars\supercite{FalconCost}. 

SpaceX souhaite \textbf{continuer sa conquête du marché} avec sa fusée \textbf{Falcon Heavy} qui dispose d'une capacité similaire à Ariane 5 pour un coût de lancement estimé à 90 millions de dollars. Alors que les premiers essais de Falcon Heavy vont commencer cet hiver\supercite{FalconHeavyFirstLaunch}, \textbf{Ariane 6 ne sera pas opérationnelle avant 2020}\supercite{Ariane6Debuts}. \textbf{Arianespace espère redevenir compétitif} avec son coût de lancement entre 60 et 115 millions d'euros (selon la configuration et les charges)\supercite{Ariane6Cost}.

Avec sa \textbf{capsule habitée Dragon 2} en développement, l'entreprise souhaite également \textbf{récupérer le marché du transport d'astronautes} pour la Station Spatiale Internationale qui dépend actuellement des capsules Soyouz ; et l'envoyer \textbf{effectuer ses premières missions sur Mars} d'ici la fin de la décennie\supercite{ElonMuskMars2020}.

\vspace{5mm}

Pour sa part Tesla Motors \textbf{a abordé le marché par le haut} en débutant avec des modèles très haut de gamme puis en descendant de gamme \textbf{à fur et à mesure que la technologie est maîtrisée et abordable}. L'entreprise suit scrupuleusement son « plan secret »\supercite{TeslaSecretMasterPlan} depuis dix ans : tandis que le Roadster était commercialisé à 90 000 dollars, les très luxueux modèles S et X commencent à 60 000 dollars et 83 000 dollars respectivement et \textbf{le futur Model 3 devrait débuter à 35 000 dollars}.

Ce Model 3 va s'avérer être \textbf{le modèle de la maturité} pour Tesla : l'entreprise devra produire plusieurs centaines de milliers d'exemplaire d'ici 2017, soit \textbf{bien plus que sa capacité de production actuelle} de 107 000 exemplaires en 2015\supercite{Tesla2015FullYearUpdate}. Cette quantité d'exemplaire va également nécessiter une \textbf{quantité de batteries sans précédent} puisque selon les dires de Musk\supercite{Model3Unveil} le double de la production actuelle de batteries sera nécessaire.

Pour y remédier, Tesla Motors a ouvert une seconde usine en Europe à \textit{Tilbourg} aux Pays-Bas en 2013\supercite{TeslaTilburgFactory} et sa \textbf{Gigafactory}, une gigantesque usine de batteries recouverte de panneaux solaires au plein milieu du Nevada, fondée en collaboration avec \textit{Panasonic}. \textbf{Capable de produire 35 millions de mWh} de batteries, elle devrait commencer la production dès 2017\supercite{Gigafactory}.

Enfin, Tesla \textbf{doit parvenir à devenir profitable} puisqu'en 2015 l'entreprise perdait encore 4000 dollars sur chaque Model S produite\supercite{TeslaLoses}.

\vspace{5mm}

Bien que son énorme catalogue et sa facilité d'utilisation aient participé au succès de Spotify auprès des utilisateurs, \textbf{il lui faut rester attrayant auprès des maisons de disques}. D'un côté se trouvent les cinq principales maisons de disques qui possèdent 18\% des parts de l'entreprise\supercite{LabelsOwnSpotify}. De l'autre se trouvent les labels indépendants qui \textbf{suspectent que leurs revenus soient inférieurs} à ceux des labels actionnaires\supercite{SpotifyBloomberg}.

Pour attirer ces derniers, Spotify leur \textbf{promet des revenus sur le long-termes} puisqu'ils sont récurrents et générés à chaque écoute, là où un album ne rapporte que lors de sa vente. De plus Spotify leur \textbf{fourni de nombreuses informations sur leur audience}, permettant par exemple de déterminer les lieux idéaux pour la tournée d'un groupe.

L'avenir de Spotify s'annonce difficile puisque l'entreprise \textbf{peine à être rentable} à cause des sommes astronomiques qu'elle doit reverser aux labels\supercite{SpotifyChiffres2015}. De plus l'arrivée d'Apple sur le marché risque de ne pas arranger ses affaires puisque son service \textbf{Apple Music a recruté 15 millions d'abonnés en tout juste un an}\supercite{AppleMusic15Million} tandis que Spotify n'en a recruté que 10 millions sur la même période.

\vspace{5mm}

Alors que Netflix a su attirer ses premiers utilisateurs grâce à son catalogue, \textbf{il a fallu qu'il puisse les retenir}, et par extension, garder ses revenus. Pour cela\textbf{ Netflix a su se rendre indispensable grâce à son système de recommandations} intégré à son service de location de DVD puis à son service de vidéo à la demande.

À partir de l'analyse des films et séries par des employés indépendants et l'\textbf{analyse des habitudes de visionnage} de ses utilisateurs, Netflix est capable d'effectuer des recommandations en fonction de l'historique de l'utilisateur et du contexte comme l'heure de la journée\supercite{NetflixAlgorithm}.

Mais plus encore, Netflix a su tirer profit de ces analyses afin de \textbf{commander des séries exclusives et acclamées par la critique capables d'attirer de nouveaux utilisateurs}. On peut citer des séries telles que \textit{House of Cards}\supercite{HoCAwards}, \textit{Orange Is The New Black}\supercite{OitNBAwards} et \textit{Narcos}\supercite{NarcosAwards}. Les saisons de ces séries ont également la particularité, contrairement au modèle de diffusion classique, d'être disponibles dans leur intégralité, laissant le choix aux utilisateurs de les regarder au rythme qu'ils souhaitent.

\textbf{L'objectif affiché depuis quelques années est désormais le développement à l'international} : le service de vidéo à la demande a commencé sa conquête de l'Europe en 2012\supercite{NetflixEurope} et est désormais disponible dans le monde entier depuis le mois de Janvier\supercite{NetflixAvailableWorld}. Enfin, Netflix compte produire du contenu local tels que la série \textit{Marseille} afin d'attirer les clients de ces nouveaux pays\supercite{NetflixNumerama}.

\vspace{5mm}

En conclusion ces entreprises \textbf{ont su tirer profit de nouveaux espaces de marché avec des produits novateurs, des stratégies en rupture et des cultures d'entreprise fortes} leur offrant rapidité, réactivité et agilité \textbf{afin de croître rapidement et de s'attaquer par la suite aux géants}.

Bien que ce mémoire prenne pour exemples Netflix, Spotify, SpaceX et Tesla Motors, ce ne sont pas les seules entreprises qui ont usé d'une stratégie et d'une culture d'entreprise similaires. Parmi les exemples les plus parlants, on peut citer \textbf{Zappos}.

\vspace{5mm}

Zappos est un \textbf{service de vente de chaussures en ligne créé en 1998} à partir du constat simple de l'entrepreneur \textit{Nick Swinmurn} : le marché des chaussures représente près de 40 milliards de dollars et il souhaite avoir sa part du marché.

Cependant \textbf{personne ne pense que les clients sont prêts à acheter des chaussures sans les essayer}. Beaucoup d'investisseurs trouveront cette idée risquée, jusqu'à ce qu'il rencontre \textit{Tony Hsieh}, convaincu par les 5\% de parts de marché représentés par la vente par correspondance.

Outre la création du marché de la vente de chaussures en ligne, Zappos \textbf{définira sa culture autour du service rendu et de la satisfaction du client}, imposant à ses employés de faire tout ce qui en leur pouvoir pour résoudre les problèmes de leurs clients, quitte à les diriger vers des sites concurrents en cas de rupture de stock.

Désormais une filiale d'\textit{Amazon} suite à son rachat en 2009 pour 1,2 milliards de dollars et \textbf{réalisant près de 2 milliards de dollars de bénéfices par an}\supercite{ZapposForbes} ; l'entreprise \textbf{se dirige vers une organisation holacratique}, une structure horizontale et auto-organisée \textbf{considérée par Tony Hsieh comme l'avenir des organisations d'entreprises.}