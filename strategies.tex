\chapter{Des stratégies en rupture avec le marché existant}

\section{Rendre l'exploration spatiale abordable}

Après avoir été écarté de la direction de \textit{X.com} ---l'entreprise qu'il a co-fondé dans l'espoir de révolutionner les paiements en ligne et qui deviendra plus tard \textit{PayPal}--- \textit{Elon Musk} se met en tête de réaliser un de ses rêves de toujours : \textbf{coloniser Mars}.

Après s'être aperçu avec le plus grand désarroi que la conquête de la planète rouge n'était pas à l'agenda de la NASA, il rejoint la \textit{Mars Society}, une organisation à but non-lucratif qui promeut l'exploration et la colonisation de la planète. Dès lors il affiche son souhait de participer à cette aventure en \textbf{lançant lui-même une campagne d'exploration spatiale}, un projet que de nombreux millionnaires avant lui ont entreprit avant d'échouer ou d'abandonner.

Il se tourne d'abord vers des fournisseurs russes dans l'espoir de racheter et réhabiliter des missiles intercontinentaux de l'ex-URSS, mais ces derniers étant trop chers à son goût, \textbf{il décide de créer ses propres fusées}.

\vspace{5mm}

Sa stratégie est alors très claire : alors que \textbf{la technologie des fusées n'a pas évolué depuis près de cinquante ans} et que le \textbf{marché est concentré autour de peu d'acteurs} s'entêtant à créer des fusées excessivement puissantes et chères, son objectif est de capitaliser sur les avancées de la recherche sur les matériaux et les simulations physiques afin de \textbf{produire des fusées abordables capables d'emporter de petites charges} tout en évitant un maximum de gâchis.

Avec cette stratégie, il souhaite ainsi \textbf{relancer l'intérêt pour la conquête spatiale en la rendant plus abordable}, et par la même occasion \textbf{explorer le marché encore délaissé de l'envoi de petites charges} à peu de frais.

Il s'entoure alors de \textit{Tom Hueller}, un ingénieur spécialisé dans les moteurs de fusées depuis près de vingt ans ; et \textit{Chris Thompson}, qui géra la production des fusées Delta et Titan chez Boing. Il fonde \textit{Space Exploration Technologies} ---souvent désigné sous le nom \textbf{SpaceX}--- en Juin 2002 et lance le développement de la fusée \textbf{Falcon 1}, une fusée à deux étages capable d'emporter une charge pouvant aller jusqu'à 670 kg.

\vspace{5mm}

La tâche se révèle cependant difficile, d'autant plus que le planning prévu par Elon Musk était très optimiste. Après trois essais ratés, \textbf{SpaceX réussira à envoyer leur première charge en orbite le 28 septembre 2008}.

\vspace{5mm}

Dès lors l'entreprise va réaliser des \textbf{lancements réguliers afin de s'assurer une indépendance financière} et commence à s'intéresser aux marchés plus importants avec des fusées de plus grandes capacités : la fusée \textbf{Falcon 5} dont le premier étage devait posséder 5 moteurs et capable d'emporter jusqu'à 4100 kg, rapidement abandonnée au profit de la fusée \textbf{Falcon 9} et du vaisseau \textbf{Dragon} développés conjointement dans l'objectif de remplacer la navette spatiale américaine et ainsi profiter du juteux contrat de ravitaillement de la Station Spatiale Internationale.

\vspace{5mm}

Alors que les principaux constructeurs de fusées se contentaient de laisser les différents étages de leurs fusées se noyer dans l'océan et de reconstruire une nouvelle fusée pour chaque lancement, Musk ambitionne de rendre le premier étage de cette nouvelle \textbf{fusée réutilisable dans un esprit de diminuer encore et toujours les coûts}\supercite{MuskAmbitionReusableFalcon9}.

\textbf{L'économie envisagée est alors de 30\%} puisque la remise en état et le carburant ne coûteraient que 4 millions de dollars alors que le premier étage coûte 60 millions de dollars à construire\supercite{SpaceXReusable30Percent}.

\section{Une voiture électrique pour tous}

\textit{Martin Eberhard} et \textit{Marc Tarpenning} sont deux ingénieurs en électronique qui ont créé la technologie des premiers livres électroniques. Après le rachat de leur entreprise, Eberhard commence à chercher une alternative à l'essence et après avoir étudié les piles à combustible, il se tourne vers \textbf{les batteries Lithium-Ion qui ont réalisé des progrès important en terme de capacité} en ce début de vingt-et-unième siècle.

Il se met alors à \textbf{étudier le marché}, réalise des calculs et décide de réaliser une voiture de sport haut-de-gamme, légère et électrique basée sur le châssis d'une Lotus Elise et \textbf{visant les millionnaires souhaitant s'afficher au volant d'une voiture rapide et écologique}.

Il fait également le choix de vendre les voitures directement par Internet et des agences luxueuses dans les centres commerciaux puisqu'ils n'ont pas besoin d'avoir des centaines de concessions, les véhicules électriques nécessitant peu d'entretien.

\vspace{5mm}

Avec Tarpenning ils fondent \textbf{Tesla Motors} en Janvier 2004 et cherchent à lever 7 millions de dollars. Ils rencontrent \textit{Elon Musk} ---qui a reçu 165 millions de dollars suite au rachat de Paypal par eBay--- qui désire changer la dépendance de la société au pétrole avec des véhicules électriques depuis le lycée. \textbf{Il investira 6,5 millions de dollars}, permettant ainsi à Tesla Motors de débuter la conception de leur premier modèle, la \textbf{Tesla Roadster}. Prévu pour début 2006 au prix de 90 000 dollars, il est doté d'une autonomie de 250 milles, une première.

Après plusieurs retards, problèmes de production, réorganisations internes et être passé non loin de la faillite, \textbf{le premier exemplaire de la Roadster sortira des chaînes de production en Février 2008}.

\vspace{5mm}

L'arrivée de liquidités et de nouveaux investisseurs permettra à Tesla de passer à la seconde phase de son plan\supercite{TeslaSecretMasterPlan} et de commencer le développement d'\textbf{une berline de luxe visant les familles aisées et prescriptrices} : la Model S, dont les premiers modèles seront livrés en Juin 2012.

Alors que le Roadster était avant tout une preuve de concept basé sur un châssis existant et assemblé par Lotus\supercite{LotusPosition}, \textbf{ce nouveau modèle serait entièrement conçu et assemblé par Tesla} ---du châssis à la carrosserie--- dans une nouvelle usine à Fremont, Californie.

Cette dernière est rachetée en 2010 à \textit{NUMMI} ---une co-entreprise de \textit{General Motors} et \textit{Toyota} qui n'a pas survécu aux difficultés du marché automobile américain suite à la crise économique de 2008--- et \textbf{est entièrement réorganisée sur le modèle de l'usine SpaceX} avec des chaînes automatisées.

\vspace{5mm}

Le Model S est devenu en très peu de temps \textbf{une icône des véhicules électriques} et est même parvenu à devenir \textbf{le véhicule de luxe le plus vendu} aux États-Unis\supercite{Tesla2015FullYearUpdate} et en Europe\supercite{ModelSLuxuryCarEurope} en 2015.

Bien évidemment, Tesla n'a pas uniquement compté sur son arrivée anticipé sur le marché des véhicules électriques pour arriver à ces résultats : \textbf{la communication est très centrée autour de la sécurité} ---la Model S fait partie des rares voitures à avoir cinq étoiles aux tests de sécurité routière américain \textit{NHTSA} et européen \textit{NCAP}\supercite{ModelSNCAP}---, l'entreprise allant jusqu'à rappeler des milliers de modèles présentant un défaut\supercite{ModelSPartialRecall}, à proposer la pose gratuite d'un bouclier en titane pour protéger les batteries\supercite{ModelSTitaniumShield} et à communiquer sur divers incidents mitigés par la conception de la voiture\supercite{ModelSFire}\supercite{ModelSAccident}.

\vspace{5mm}

Avant même que le Model S sorte ---une habitude de Musk--- Tesla annonce le 9 février 2012 \textbf{un nouveau modèle encore plus attrayant pour les familles} : le Model X, un monospace-SUV de luxe \textbf{basé sur le châssis du Model S} capable de transporter jusqu'à 7 adultes grâce à la place gagnée par l'absence d'un moteur thermique et la présence de portes « \textit{Falcon Wing} » permettant de pouvoir s'asseoir aisément dans la troisième rangée de siège. Les premiers modèles ont été livrés en Amérique du Nord en Septembre 2015 tandis que les livraisons européennes débutent en Juin 2016.

\section{Un service plus pratique que le téléchargement illégal}

Été 2006 en Suède, tous les regards sont tournés vers le procès qui oppose \textit{The Pirate Bay} à l'industrie de la musique. À ce même moment, \textit{Daniel Ek}, jeune entrepreneur et président de \textit{uTorrent} ---l'entreprise éditant l'un des principaux clients BitTorrent, protocole utilisé par The Pirate Bay pour mettre à disposition des fichiers soumis au droit d'auteur---, \textbf{souhaite réaliser un service aussi pratique que \textit{Napster}} ---service qu'il utilisait adolescent et qui a fermé suite au procès subit en 2000--- mais ayant \textbf{un business model viable et capable de rémunérer les artistes}. 

Pour Ek, \textbf{le téléchargement illégal est compliqué} : trouver le titre qu'on souhaite est pénible, le téléchargement dure plusieurs minutes et il faut s'inquiéter des virus informatiques. On ne télécharge non pas par choix, mais par nécessité et Ek est convaincu que si un service est plus pratique et propose une meilleure expérience que le piratage, alors \textbf{les utilisateurs seront prêts à payer pour ce service}.

\vspace{5mm}

Ek fonde alors \textbf{Spotify AB} en emmenant avec lui \textit{Ludvig Stigeus} ---créateur de uTorrent--- avec l'objectif de créer ce service.

Stigeus va se révéler important pour l'aspect technique de Spotify, puisque Ek souhaite \textbf{pouvoir écouter n'importe quel titre en moins de 200 millisecondes}, ce qui était alors inconcevable à l'époque. Il va réaliser l'impossible en tirant parti du protocole BitTorrent afin de pouvoir récupérer rapidement les premières secondes d'un titre depuis l'ordinateur d'un autre utilisateur avant de récupérer le reste du titre à partir des serveurs de Spotify.

L'autre désir de Ek se trouve dans le partage des playlists : lorsqu'il utilisait Napster, il avait pour habitude de découvrir de nouveaux artistes en trouvant des utilisateurs aux goûts similaires puis en téléchargeant leur libraire musicale. \textbf{Ces playlists sont désormais un atout majeur de Spotify} qui possède plusieurs dizaines d'employés qui alimentent des centaines de playlists par mois.

Une fois l'aspect technique réglé, Spotify pu se concentrer sur les accords avec les maisons de disque, ce qui ne fut pas aisé : \textbf{les négociations ont duré près de deux ans}, d'autant plus que les fondateurs n'avaient aucune relations avec les ayants droit.

\vspace{5mm}

Le lancement de Spotify en Octobre 2008 s'avère être un succès immédiat puisqu'il a \textbf{réussi l'exploit de recruter près de 17\% de la population suédoise}\supercite{IFPIDRM2010}, pour laquelle le téléchargement illégal était devenu une habitude.

Après un lancement aux États-Unis en 2011 puis dans le reste du monde, Spotify annoncera en Juin 2015 avoir dépassé les \textbf{30 millions de titres}\supercite{SpotifyPSMusic},  \textbf{être passé de 10 à 20 millions d'abonnés en un an} et \textbf{reversé 3 milliards de dollars de royalties}\supercite{Spotify20Million}. La croissance de Spotify ne semble pas s'essouffler puisque Daniel Ek annoncera avoir atteint les \textbf{30 millions d'utilisateurs payants} le 21 Mars 2016\supercite{Spotify30Million}.

\section{Le service incontournable de location de films et séries}

Après avoir revendu sa société Pure Software en 1997, \textit{Reed Hastings} loue une copie de \textit{Apollo 13} chez Blockbuster. Cependant après six semaines, la boutique lui demande de payer près de soixante dollars de frais de retard. Alors qu'il est à la salle de sport, il s'aperçoit que cette dernière a un meilleur business model que Blockbuster : « You could pay \$30 or \$40 a month and work out as little or as much as you wanted. »\supercite{NetflixOrigins}

Il s'associe avec \textit{Marc Randolph} et investit la somme du rachat de Pure Software pour créer \textbf{Netflix} qui \textbf{se différencie de ses concurrents} par une \textbf{formule par abonnement}, la location par correspondance \textbf{via Internet} sans aucune boutique et la\textbf{ possibilité de conserver une copie aussi longtemps que souhaité} (cependant le nombre de copies pouvant être louées à un même moment est limité). 

\vspace{5mm}

Netflix fait un pari majeur en choisissant d'être \textbf{le premier à proposer la location de DVD} alors que le support n'est disponible que depuis seulement deux ans aux États-Unis et \textbf{commande l'intégralité des films et séries disponibles en DVD afin de rendre le service incontournable}\supercite{NetflixHistory}. Le pari s'avérera payant lorsque les lecteurs DVD deviendront populaires à Noël en 2001\supercite{BBC_DVD}, augmentant mécaniquement la popularité de Netflix qui enverra en 2005 \textbf{jusqu'à un million de DVD par jour parmi un catalogue de 35 000 titres}\supercite{NetflixHistory}.

De plus dès ses débuts l'entreprise a souhaité \textbf{connaître au mieux les goûts de ses clients} à partir des statistiques des titres loués \textbf{afin de proposer les références qui sont le plus en adéquation avec leurs attentes}. Netflix a par la suite développé le service de recommandation \textbf{CineMatch} et l'a présenté comme l'un de ses principaux atouts.

\vspace{5mm}

Toujours en avance sur son temps, Reed Hastings voit le potentiel de la vidéo à la demande et juge que \textbf{la dématérialisation pourrait diminuer drastiquement} les coûts liés à l'achat des DVD, aux entrepôts et aux envois postaux.

Netflix lance alors son service de vidéo à la demande illimitée sur PC en complément de leur service de location de DVD en 2007. \textbf{Grâce à son avance, son système de recommandation, des partenariats clés} comme Microsoft et Apple pour proposer son service sur leurs appareils et la \textbf{production de séries exclusives extrêmement populaires}, le service atteint en 2014 les \textbf{35 millions d'abonnés et près de 240 millions de dollars de profits}\supercite{NetflixDVDBloomberg}.