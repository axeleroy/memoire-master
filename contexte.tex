\chapter{Contexte}

Afin de mettre en contexte la réussite des entreprises étudiées, nous allons nous pencher sur l'état de leurs marchés respectifs avant leur création durant le début des années 2000.

\section{Un marché de la location de films dominé par Blockbuster}

Sur le marché américain l'entreprise \textit{Blockbuster} fait figure de poids lourd dans le marché de la location de films depuis près de quinze ans mais se voit concurrencé par la jeune société \textit{Netflix} qui ---contrairement à Blockbuster--- n'impose pas de délais de retour pour les DVD et ne possède aucune boutique physique. 

\section{Une industrie aérospatiale au point mort depuis la fin de la Guerre Froide}

La chute de l'\textit{URSS} en 1991 a marqué la fin de la Guerre Froide et des ambitions de conquête spatiale des États-Unis : bien que la \textit{NASA} a lancé le programme de la \textit{Station Spatiale Internationale} dès 1993, elle ne souhaite plus réaliser de missions spatiales habitées au-delà de la stratosphère, et ce par manque de budget alloué par le Congrès américain. 

De plus le marché de l'envoi de satellites est restreint à une poignée d'entreprises et d'organisations en position de monopole : le français \textit{Arianespace}, les américains \textit{Lockheed Martin Space Systems} et \textit{Boeing} et le russe \textit{RSC Energia} qui pratiquent des prix dissuasifs pour les petites entreprises et organisations.

\section{Des véhicules électriques qui manquent d'autonomie}

On peut difficilement parler d'une industrie tant elle est à ses balbutiements : les principaux constructeurs automobiles ne daignent pas s'intéresser à la création de véhicules électriques\footnote{Exception faite de \textit{Nissan} qui a produit la \textit{Hypermini} ---produite de 1999 à 2001 et disposant d'une autonomie de 115 kilomètres--- destinée uniquement au marché nippon et la \textit{Altra EV} ---une conversion de la \textit{R'nessa} produite de 1997 à 2001 et disposant d'une autonomie de 230 kilomètres--- qui est restée confiné à un programme pilote comprenant 200 exemplaires. Nissan attendra 2010 pour rentrer à nouveau dans le marché avec sa \textit{Leaf}.} et seuls une poignée de groupes d'étudiants et quelques entreprises de taille mineure ont exploré la création de véhicules entièrement électriques.

Ils se heurtent cependant à un problème majeur : les technologies de batteries de l'époque ne permettent pas de proposer une autonomie décente.

\section{Un marché de la musique cannibalisé par le téléchargement illégal}

Après avoir culminé à 38 milliards de dollars en 1999\supercite{IFPI2000Report}, les revenus de l'industrie de la musique n'ont cessé de chuter jusqu'à atteindre les 18 milliards de dollars en 2008\supercite{IFPI2008Report}. L'industrie accuse alors le téléchargement illégal d'en être à la cause et attaque en justice l'entreprise éditant le logiciel \textit{Napster} en 2000 et la plate-forme de recherche BitTorrent \textit{The Pirate Bay} en 2006, forçant l'un à fermer et l'autre à fuir continuellement les autorités. 

\textit{iTunes} ---la boutique de musique dématérialisée d'Apple--- bien qu'ayant apporté la possibilité d'acheter les titres à l'unité, ne permet cependant pas à l'industrie de combler la chute des ventes de disques. Les principales maisons de disque sont alors désespérées.