\chapter{Introduction}

Alors que j'ai toujours été fasciné par l'entrepreneuriat et l'innovation qui en découle, ce début d'année 2016 a été marqué par la concrétisation des efforts de plusieurs entreprises clés, ce qui confirme mon sentiment que ces entreprises innovantes changent notre monde :

\vspace{5mm}

L'entreprise \textbf{SpaceX} a réussi le 27 mai dernier à récupérer le premier étage de sa fusée Falcon 9 pour la troisième fois consécutive\supercite{SpaceXThirdLanding} et a ainsi confirmé la viabilité de son projet de fusées réutilisables. Ce après avoir rendu l'exploration spatiale abordable, créé un marché pour l'envoi de charge de poids réduit et être devenue la première entreprise privée à ravitailler la Station Spatiale Internationale le 25 mai 2012\supercite{DragonFirstISSMission}. Pourtant cette entreprise qui souhaite réaliser des voyages habités sur Mars d'ici 2025\supercite{ElonMuskMars2020} n'avait aucune expérience dans le domaine aérospatial lors de sa création en 2002.

\vspace{5mm}

Le 10 février \textbf{Tesla Motors} annonce avoir écoulé 107 000 exemplaires de leurs modèles de voitures électriques S et X en trois ans et demi\supercite{Tesla2015FullYearUpdate}. L'entreprise  qui a commencé avec des passionnés d’électronique rêvant de changer le monde a par la suite dévoilé le 31 mars son quatrième modèle ---le bien nommé \textit{Model 3}. Ce modèle destiné à démocratiser les véhicules électriques auprès des classes moyennes cumulait au 4 mai près de 325 000 de réservations\supercite{Tesla2016Q1Update}, pour une livraison fin 2017 en Amérique du Nord.

\vspace{5mm}

Les 11 et 12 avril \textit{Reed Hastings} présentait la stratégie de sa société \textbf{Netflix} pour les années à venir à la Cité du Cinéma, renommée \textit{Netflix City} pour l'occasion. Le symbole est fort et à l'image des changements que le service a apporté à l'industrie du divertissement et à notre manière de consommer les films et séries. Tout d'abord créé suite au mécontentement de son créateur à l'égard du service de location de films Blockbuster, alors en position de monopole, il a su se faire une place et créer par la suite un nouveau marché avec son service de vidéo à la demande.

\vspace{5mm}

\textbf{Spotify} quand à eux ont annoncé des revenus records de 1,95 milliards d'euros pour l'année 2015 \supercite{SpotifyChiffres2015}. Créé sur le postulat qu'il fallait proposer un service légal qui soit plus pratique que le téléchargement illégal dans une Suède qui avait alors les yeux braqués sur le procès \textit{The Pirate Bay}, le service a su détourner une génération entière du téléchargement illégal et générer plus de revenus que la vente de disques et le téléchargement légal.

\vspace{5mm}

Ces entreprises ont la particularité d'avoir moins d'une vingtaine d'années d'existence et alors qu'elles n'avaient aucune expérience dans leurs domaines respectifs, ont su devenir pionnières de leurs marchés respectifs et sont allés jusqu'à entrer en compétition avec des acteurs historiques.

\textbf{Elles ont pour point commun d'être en rupture totale avec leurs marchés respectifs tout comme sur leurs manières de développer leurs cultures d'entreprise.}